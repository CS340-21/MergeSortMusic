% This document was built with Overleaf and should be edited from overleaf. 
% Merge Sort Music Project proposal v1 Febuary 8th 2021

\documentclass{article}

% If you're new to LaTeX, here's some short tutorials:
% https://www.overleaf.com/learn/latex/Learn_LaTeX_in_30_minutes
% https://en.wikibooks.org/wiki/LaTeX/Basics

% Formatting
\usepackage[utf8]{inputenc}
\usepackage[margin=1in]{geometry}
\usepackage[titletoc,title]{appendix}

% Math
% https://www.overleaf.com/learn/latex/Mathematical_expressions
% https://en.wikibooks.org/wiki/LaTeX/Mathematics
\usepackage{amsmath,amsfonts,amssymb,mathtools}

% Images
% https://www.overleaf.com/learn/latex/Inserting_Images
% https://en.wikibooks.org/wiki/LaTeX/Floats,_Figures_and_Captions
\usepackage{graphicx,float}

% Tables
% https://www.overleaf.com/learn/latex/Tables
% https://en.wikibooks.org/wiki/LaTeX/Tables

% Algorithms
% https://www.overleaf.com/learn/latex/algorithms
% https://en.wikibooks.org/wiki/LaTeX/Algorithms
\usepackage[ruled,vlined]{algorithm2e}
\usepackage{algorithmic}

% Code syntax highlighting
% https://www.overleaf.com/learn/latex/Code_Highlighting_with_minted
\usepackage{minted}
\usemintedstyle{borland}

% Title content
\title{Merge Sort Music Preliminary Proposal}
\author{Georgia Stricklen -- Jonathan Ting -- John Carmack -- Shreyank Patel}
\date{Febuary 8, 2021}

\begin{document}

\maketitle

% Introduction
\section{Introduction}  
  
While many media hosting services do allow users to manage and organize their personal content into playlists, the functionality of such services can be lacking in both efficiency and efficacy. 
When users of these services wish to partake in the media offered, rarely do they wish to spend time battling the cumbersome user interface for building playlists that the services offer. In addition, many of these services lack the ability to sort the playlist of media on criteria that many users find significant (such as user rating).   

We propose a web application that will allow users to import libraries of media from these media services and then organize them into playlists based on user-generated criteria. These playlists will then be able to be exported to the media services that they came from. Users will also be able to store these playlists with their account such that it will serve as a backup external to the media hosting platform that it’s for. As far as the team is aware there are already a number of similar applications that provide this service, but they offer it for specific, obscure media hosting services. Our proposed application would offer similar services for Spotify, which is a far more mainstream service.  
Jonathan Ting is a senior studying Computer Engineering currently doing undergraduate research in hardware security. However, as a hobby, he also enjoys dabbling in web development, both with backend frameworks such as Django and frontend frameworks such as Bootstrap. This project will fulfill a wish of his, to optimize the process of sorting music playlists, of which there are not many great tools for at the moment.  

Shreyank Patel is a senior majoring in Computer Science and Economics. He enjoys running, and cooking. His areas of interest include Machine Learning and Cryptography. He has prior experience designing websites using HTML, CSS, Javascript and React. Through this project he hopes to gain meaningful experience in Bootstrap, Django and other frameworks.
Georgia Stricklen is a Junior at the University of Tennessee, majoring in Computer Science. Her main focus is C++ and Java with some experience in Python. She enjoys reading and video games. Solving real world problems while bettering existing structures is what she hopes to work towards in her career as a Computer Scientist. 

John Carmack is a Senior majoring in computer science. Web development is a primary interest for him, though he has only ever worked on fairly static web sites with limited user interactivity. He enjoys rock climbing, brazillian jiu-jitsu and works as a commissioned artist in his spare time.

%  Customer Value
\section{Customer Value}
These are the motivations for our proposed web application and our projected outcomes for users. 
\subsection{Need}
Spotify occupies a significant portion of the music media hosting market and they serve millions of customers a day. As highlighted by many members of the team, while the hosting service does offer a user interface for managing playlists, it can be cumbersome and does not fully support all sorting criteria that the team feels is important using the service to the fullest extent. It is our belief that many other users of Spotify share our sentiments, and would relish a more engaging and granular playlist building experience. 
\subsection{Solution}
Our application would provide users with a personalized web application that allows them to import, create, edit, sort, and export playlists to Spotify. The user’s account will also provide the secondary benefit of providing a backup of their playlist so that should they inadvertently edit or delete the playlist, they can simply restore it via our web service. 

We would consider this application a success if users were able to say definitively that the user interface provided by our service is clearly superior to Spotify’s, and that it contributes to the fulfillment they get from using the media hosting service. We wish to both streamline and provide increased customizability to Spotify’s playlist feature such that the user’s experience is enhanced.
\subsection{System}
Our software will be a fully-fleshed RESTful web application. We will require a back end that can interface with the user via a website, the Spotify API, and a database to store user information. From the user’s perspective we will require a front-end that is both intuitive and simple while also providing the user with the rich functionality that we find Spotify’s playlist manager lacking. We propose that this is the lightest-weight system architecture that can accomplish the goals we have in mind. During development we will test the application’s functionality via dummy accounts. Some possible enhancements that could be a value to the user would be the ability to save the playlist and export it to the service that provided the original playlist.

% Technology & Tools
\section{Technology \& Tools}
Our front-end will be composed of the big three webtools: HTML, CSS, and Javascript.
We are also using the react framework to help organize and streamline the front-end creation process.  

Our back end is being implemented in python’s Django framework, which uses SQLite as its default database. We have as of yet to decide to keep SQLite as our database technology for deployment.  

The IDE of choice for the team seems to be VS code thanks to the myriad of tools available to the django and REACT developer.

% Team
\section{Team}
These assignments are relatively flexible and subject to the requirements of the project. 
\subsection{Georgia Stricklen}
Merge Sort Algorithm and back end.
\subsection{Jonathan Ting}
Django Database Design and Backend API
\subsection{John Carmack}
Front end design and development. Documentation
\subsection{Sheyank Patel}
Front end design and development. 

% Team Management
\section{Team Management}
We will have weekly discord meetings to discuss project progress and bring up issues. 
\subsection{Schedule}
\subsubsection{Week 1}
Skeleton code for front and back end.
\subsubsection{Week 2}
Proposal draft
\subsubsection{Week 3}
Backend: Import Spotify playlists -- Frontend: Rough draft layout
\subsubsection{Week 4}
Backend: Sorting algo -- Frontend: Tentative backend integration.  
\subsubsection{Week 5}
Backend: login -- Frontend: CSS/MaterialUI or Bootstrap
\subsubsection{Week 6}
Backend: optimizations -- Frontend: Revision / Upgrading
\subsubsection{Week 7}
Collaborative sorting
\subsection{Constraints}
We anticipate no regulatory or ethical constraints. Users will not be sharing sensitive data with out application. 
\subsection{Resources}
All data will either be provided by users or via the Spotify API. We anticipate no difficulty in procuring said data.
\subsection{Descoping}
Our project is limited in scope and we fully expect to be able to complete all schedule items on time. 
If that proves to not be the case, however, It should be fully feasible to provide users with the full functionality of editing existing Spotify playlists.
\end{document}
